% Options for packages loaded elsewhere
\PassOptionsToPackage{unicode}{hyperref}
\PassOptionsToPackage{hyphens}{url}
%
\documentclass[
]{article}
\usepackage{lmodern}
\usepackage{amssymb,amsmath}
\usepackage{ifxetex,ifluatex}
\ifnum 0\ifxetex 1\fi\ifluatex 1\fi=0 % if pdftex
  \usepackage[T1]{fontenc}
  \usepackage[utf8]{inputenc}
  \usepackage{textcomp} % provide euro and other symbols
\else % if luatex or xetex
  \usepackage{unicode-math}
  \defaultfontfeatures{Scale=MatchLowercase}
  \defaultfontfeatures[\rmfamily]{Ligatures=TeX,Scale=1}
  \setmainfont[]{Source Sans Pro}
\fi
% Use upquote if available, for straight quotes in verbatim environments
\IfFileExists{upquote.sty}{\usepackage{upquote}}{}
\IfFileExists{microtype.sty}{% use microtype if available
  \usepackage[]{microtype}
  \UseMicrotypeSet[protrusion]{basicmath} % disable protrusion for tt fonts
}{}
\makeatletter
\@ifundefined{KOMAClassName}{% if non-KOMA class
  \IfFileExists{parskip.sty}{%
    \usepackage{parskip}
  }{% else
    \setlength{\parindent}{0pt}
    \setlength{\parskip}{6pt plus 2pt minus 1pt}}
}{% if KOMA class
  \KOMAoptions{parskip=half}}
\makeatother
\usepackage{xcolor}
\IfFileExists{xurl.sty}{\usepackage{xurl}}{} % add URL line breaks if available
\IfFileExists{bookmark.sty}{\usepackage{bookmark}}{\usepackage{hyperref}}
\hypersetup{
  pdftitle={Using pins},
  pdfauthor={Your Name},
  hidelinks,
  pdfcreator={LaTeX via pandoc}}
\urlstyle{same} % disable monospaced font for URLs
\usepackage[margin=1in]{geometry}
\usepackage{graphicx,grffile}
\makeatletter
\def\maxwidth{\ifdim\Gin@nat@width>\linewidth\linewidth\else\Gin@nat@width\fi}
\def\maxheight{\ifdim\Gin@nat@height>\textheight\textheight\else\Gin@nat@height\fi}
\makeatother
% Scale images if necessary, so that they will not overflow the page
% margins by default, and it is still possible to overwrite the defaults
% using explicit options in \includegraphics[width, height, ...]{}
\setkeys{Gin}{width=\maxwidth,height=\maxheight,keepaspectratio}
% Set default figure placement to htbp
\makeatletter
\def\fps@figure{htbp}
\makeatother
\setlength{\emergencystretch}{3em} % prevent overfull lines
\providecommand{\tightlist}{%
  \setlength{\itemsep}{0pt}\setlength{\parskip}{0pt}}
\setcounter{secnumdepth}{-\maxdimen} % remove section numbering
\usepackage{titling}
\pretitle{\begin{center}\LARGE\includegraphics[width=12cm]{protips.png}\\[\bigskipamount]}
\posttitle{\end{center}}

\title{Using \texttt{pins}}
\author{Your Name}
\date{The Date}

\begin{document}
\maketitle

\hypertarget{why-you-should-use-pins}{%
\section{Why You Should Use Pins}\label{why-you-should-use-pins}}

As a data scientist, it's likely that a recurring challenge is to easily
share content with others in your organization. One of the most
persistent problems is knowing how and where to store resources,
especially resources that can't be easily put in a database for others
to access without needing to email files back and forth - for example
model outputs, csv files, and r objects.

This is where Pins come in. RStudio developed Pins to make the process
of discovering, caching, and sharing resources simpler.

If you're always asking colleagues to download files before running your
code, or are redeploying an entire app just to update the model or data
in it, or if many pieces of your content are pointed to the same
dataset, read on.

\hypertarget{what-are-pins}{%
\section{What are Pins?}\label{what-are-pins}}

Pins are almost exactly what they sound like: a way to store and then
remotely access R and Python objects. You can take an object and ``pin''
to a number of boards, including RStudio Connect, S3, Google Cloud, even
or integrate them to your website.

Once resources are pinned, you can also discover them. Pins let you
search well-known data repositories that are guaranteed to contain
datasets, instead of searching on the internet and hoping to find what
you need, as well as enhancing the discoverability of internally-created
datasets.

The use cases are too many to count, so in the next section, you'll
learn how to pin your first resource using RStudio Connect. From there,
you can check out this website for further inspiration on how to use
Pins to make your day-to-day life easier. How to Use Pins There are four
main steps to using Pins. First, install the package, get your R session
to connect with your board, and then Pin a resource. Finally, you can
get resources. Let's walk through each step using RStudio Connect as an
example.

Before you can publish a pin to the RSC board, you have to get an API
key from Connect and save this to your sys environment variables.

In RSC, generate an API key. You can name this something like
``forusingpins''. Copy the value to the clipboard. Now in R, you have to
save this API key to your system environment variables. Since you want
this to persist and not evaporate every time R restarts, write this in
your .Rprofile. Think about if you want this to be in your Project
.Rprofile file or your home directory .Rprofile depending on whether
you'll use it just for this project, or for several.

Wherever you're setting it, you'll add a line like:
\texttt{Sys.setenv("RSC\_API\_KEY"\ =\ “\textless{}Key\ Value\textgreater{}”)}

Now you're going to register the board. This isn't initializing a new
board, it's just telling your session that there's a place you can store
resources. ``Register'' here means ``I see you and know you're there.''

Use
\texttt{pins::board\_register(\ \ \ \ \ "rsconnect",\ \ \ \ \ \ server\ =\ "https://colorado.rstudio.com/rsc/",\ \ \ \ \ key\ =\ Sys.getenv("RSC\_API\_KEY")\ \ \ \ \ )}

Now you're going to Pin your first resource. Put your object on the
rsconnect board with
\texttt{pin(my\_data,\ description\ =\ "Super\ Cool\ data\ set",\ board\ =\ "rsconnect")}

Code to retrieve the pin will now be provided on the deployed content in
RStudio Connect:
\texttt{setsData\ \textless{}-\ pin\_get("username/my\_data",\ board\ =\ "rsconnect")}

You're ready to publish your Pin. The first time you publish, it will
fail saying ``{[}Connect{]} Message: `Invalid API key, the API key is
empty.'\,'' because there is no key on the environment variable on
connect. Go put RSC\_API\_KEY in environment variables and refresh.

\end{document}
